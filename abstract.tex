\documentclass[a4j,12pt]{jarticle}
\usepackage[top=25mm,bottom=25mm,left=25mm,right=25mm]{geometry}
\pagestyle{empty}

\renewcommand{\author}[1]{\def\author{#1}}
\renewcommand{\title}[1]{\def\title{#1}}
\newcommand{\supervisor}[1]{\def\supervisor{#1}}
\newcommand{\group}[1]{\def\group{#1}}

\newcommand{\makeheader}{
\newpage
\noindent\\九州産業大学 情報科学部
\noindent\\卒業研究発表会グループ\group
\noindent\\卒業研究概要
\vskip 20mm
\begin{center}
  \title
\end{center}
\vskip 15mm
}
\newcommand{\maketailer}{
\vskip 20mm
\begin{center}
\setlength{\tabcolsep}{1mm}
\begin{tabular}[h]{p{4zw}p{1em}l}
 発 表 者&:& \author (情報科学科)\\
 指導教員&:& \supervisor
\end{tabular}
\end{center}
}
\newenvironment{jkabst}{\makeheader}{\maketailer}


\begin{document}
% title, author, supervisor, group の指定は必須
\title{スマートフォン用K'sLifeアプリ「スマト君」の開発}
\author{16JK053 NGUYEN THANH LONG}
\supervisor{下川 俊彦 教授}
\group{1}

% jkabst環境の中に、概要本文を記述する
\begin{jkabst}

  九州産業大学では、学生教育支援・事務情報システム K'sLifeを運用している。
  K'sLife は、履修登録や成績確認、連絡通知、週間スケジュールなど様々な機能がある。
  また、K'sLifeにはパソコン版のページとスマートフォン版のページがある。
  しかし、スマートフォン版のページは利便性が低くあまり使われていない。

  K'sLife はスマートフォンで使用するときも、毎回ログインしなければならない。
  パソコン版のページとスマートフォン版のページがあるが、スマートフォン版の操作はステップが多い。
  また、パソコン版はスマートフォンでは操作しにくい。
  すなわち、K'sLife をスマートフォンで見るのは容易ではない。
  この問題点を解決するため、本研究では専用のスマートフォン用アプリケーションを開発する。
  このシステムを「スマト君」と名付けた。

  スマト君は、K'sLife の API を提供するための 「スマト君 API Server」と、スマートフォン用アプリケーションである「スマト君アプリケーション」と、従来の K'sLife から構成される。
  スマト君API Serverは、スマト君アプリケーションから HTTP でリクエストを受け、K'sLife にアクセスし、その結果を解析し JSON データとしてアプリケーションに返す。
  スマト君アプリケーションは、スマト君 API serverを通して取得した K'sLife のデータを表示する。
  本研究ではスマト君アプリケーションは iOS 用のアプリケーションとして実装した。

  スマト君を実際に利用してもらい、アンケートで評価した。評価結果から、高評価を得た。
  しかし、スマト君の機能はまだ少ない。
  また、デザインはシンプルである。
  今後の課題は、スマト君の機能強化とデザインの修正である。


\end{jkabst}
\end{document}
